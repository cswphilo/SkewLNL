\documentclass{article}
\usepackage{amsmath}
\usepackage{amsfonts}
\usepackage{amssymb}
\usepackage{enumerate}
\usepackage{prftree}
\usepackage{ntheorem}
\usepackage{tikz-cd}
\usepackage{hyperref}
\usepackage{float}
\usepackage{graphicx}
%\usepackage{quiver}
\usepackage[all,cmtip]{xy}
\usepackage{proof}
%\usepackage{bussproofs}
%\usepackage[a4paper, total={6in, 8in}]{geometry}
\usepackage{lscape}
\theorembodyfont{}
\newtheorem{theorem}{Theorem}[subsection]
\newtheorem{corollary}[theorem]{Corollary}
\newtheorem{lemma}[theorem]{Lemma}
\newtheorem{defn}[theorem]{Definition}
\newtheorem{remark}[theorem]{Remark}
\newtheorem*{proof}{Proof : }
\newtheorem{fact}[theorem]{Fact}
\newcommand{\ldbc}{[\![}
\newcommand{\rdbc}{]\!]}
\newcommand{\tbar}{[\vec{x}/\vec{t}]}
\newcommand{\ltbar}{[\vec{x}, x/\vec{t}, x]}
%%skew calculus macros
\newcommand{\tl}{\otimes \mathsf{L}}
\newcommand{\tr}{\otimes \mathsf{R}}
\newcommand{\lright}{\multimap \mathsf{R}}
\newcommand{\lleft}{\multimap \mathsf{L}}
\newcommand{\pass}{\mathsf{pass}}
\newcommand{\unitl}{\mathsf{IL}}
\newcommand{\unitr}{\mathsf{IR}}
\newcommand{\ax}{\mathsf{ax}}
\newcommand{\ot}{\otimes}
\newcommand{\lolli}{\multimap}
\newcommand{\I}{\mathsf{I}}
\newcommand{\msfL}{\mathsf{L}}
\newcommand{\fl}{\mathsf{FL}}
\newcommand{\gl}{\mathsf{GL}}
\newcommand{\gr}{\mathsf{GR}}
\newcommand{\fr}{\mathsf{FR}}
\newcommand{\td}{\text{-}}
\newcommand{\scut}{\mathsf{scut}}
\newcommand{\ccut}{\mathsf{ccut}}
%%skew calculus macros
%%intuitionistic macros
\newcommand{\oner}{1\mathsf{R}}
\newcommand{\tms}{\times}
\newcommand{\tmsr}{\times \mathsf{R}}
\newcommand{\tmslf}{\times \mathsf{L}^{1}}
\newcommand{\tmsls}{\times \mathsf{L}^{2}}
\newcommand{\rar}{\rightarrow}
\newcommand{\rarr}{\rightarrow \mathsf{R}}
\newcommand{\rarl}{\rightarrow \mathsf{L}}
\newcommand{\wk}{\mathsf{wk}}
\newcommand{\ctr}{\mathsf{ctr}}
\newcommand{\ex}{\mathsf{ex}}
%%intuitionistic macros
\begin{document}
 \section*{Skew symmetric LNL (Cartesian and Skew symmetric monoidal)}
 Formulae and context in skew symmetric LNL:
 \begin{itemize}
   \item Intuitionistic part: $X \mid 1 \mid X \tms Y \mid X \rar Y \mid GA$, contexts $\Phi$.
   \item Skew part: $A \mid \I \mid A \ot B \mid A \lolli B \mid FX$, contexts $\Gamma$.
   \item There is no stoup formula in intuitionistic part of LNL.
   %\item $S_{s}$ denotes stoup formula in skew symmetric calculus.
 \end{itemize}
 Sequent calculus for intuitionistic part in LNL
 \begin{displaymath}
   \begin{array}{cc}
     \infer[i\td\ax]{- \mid A \vdash_{i} A}{}
     \qquad
     \infer[i\td 1\mathsf{R}]{- \mid \quad \vdash_{i} 1}{}
     \qquad
     \infer[\tmslf]{- \mid A \tms B , \Phi \vdash_{i} Z}{- \mid A,  \Phi \vdash_{i} Z}
     \qquad
     \infer[\tmsls]{- \mid A \tms B , \Phi \vdash_{i} Z}{- \mid B,  \Phi \vdash_{i} Z}
     \\
     \infer[\tmsr]{- \mid \Phi_{0} , \Phi_{1} \vdash_{i} A \tms B}{
      \deduce{- \mid \Phi_{0} \vdash_{i} X}{}
      &
      \deduce{- \mid \Phi_{1} \vdash_{i} Y}{}
     }
     \qquad
     \infer[\rarl]{- \mid X \rar Y , \Phi_{0} , \Phi_{1} \vdash_{i} Z}{
      \deduce{- \mid \Phi_{0} \vdash_{i} X}{}
      &
      \deduce{- \mid Y , \Phi_{1} \vdash_{i} Z}{}
     }
     \qquad
     \infer[\rarr]{- \mid \Phi \vdash_{i} X \rar Y}{- \mid \Phi , X \vdash_{i} Y}
   \end{array}
 \end{displaymath}
 Structural and adjunction rules\footnote{We do not need write out $\ex$ explicitly in LNL because formulae in context are always exchangeable. However, if we want to directly see adjunction between intuitionistic logic and skew non-comuutative linear logic, we have to write it explicitly.}
 \begin{displaymath}
   \begin{array}{cc}
   \infer[i\td\wk]{- \mid X, \Phi \vdash_{i} Z}{- \mid \Phi \vdash_{i} Z}
   \qquad
   \infer[i\td\ctr]{- \mid X , \Phi \vdash_{i} Z}{- \mid X , X , \Phi \vdash_{i} Z}
   \qquad
   % \infer[i\td\ex]{- \mid \Phi , Y , X \vdash_{i} Z}{- \mid \Phi , X , Y \vdash_{i} Z}
   % \qquad
   \infer[\fr]{- \mid \Phi \vdash_{s} F X}{- \mid \Phi \vdash_{i} X}
   \end{array}
 \end{displaymath}
 Sequent calculus for skew symmetric monoidal closed
 \vspace{-0.15cm}
  \begin{displaymath}
  \begin{array}{cc}
    \infer[s\td\ax]{A \mid \quad \vdash_{s} A}{}
    \qquad
    \infer[s\td\unitr]{- \mid \quad \vdash_{s} \I}{}
    \qquad
    \infer[s\td\unitl]{\I \mid \Gamma \vdash_{s} C}{- \mid \Gamma \vdash_{s} C}
    \qquad
    \infer[s\td\tl]{A \ot B \mid \Gamma \vdash_{s} C}{A \mid B , \Gamma \vdash_{s} C}
    \\
    \infer[s\td\pass]{- \mid A, \Gamma \vdash_{s} C}{A \mid \Gamma \vdash_{s} C}
    \qquad
    \infer[s\td\tr]{S_{s} \mid \Gamma_{0} , \Gamma_{1} \vdash_{s} A \ot B}{
     \deduce{S_{s} \mid \Gamma_{0} \vdash_{s} A}{}
     \qquad
     \deduce{- \mid \Gamma_{1} \vdash_{s} B}{}
    }
    \qquad
    \infer[s\td\lright]{S_{s} \mid \Gamma \vdash_{s} A \lolli B}{S_{s} \mid \Gamma , A \vdash_{s} B}
    \\
    \infer[s\td\lleft]{A \lolli B \mid \Gamma_{0} , \Gamma_{1} \vdash_{s} C}{
     \deduce{- \mid \Gamma_{0} \vdash_{s} A}{}
     \qquad
     \deduce{B \mid \Gamma_{1} \vdash_{s} C}{}
    }
  \end{array}
  \end{displaymath}
  Intuitionistic rules in skew symmetric monoidal closed calculus
  \begin{displaymath}
    \begin{array}{cc}
      \infer[s\td\tmslf]{X \tms Y \mid \Gamma , \Phi \vdash_{s} C}{X \mid \Gamma , \Phi \vdash_{s} C}
      \qquad
      \infer[s\td\tmsls]{X \tms Y \mid \Gamma , \Phi \vdash_{s} C}{Y \mid \Gamma , \Phi \vdash_{s} C}
      \qquad
      \infer[s\td\rarl]{X \rightarrow Y \mid \Gamma , \Phi \vdash_{s} C}{
        \deduce{- \mid \Phi \vdash_{i} X}{}
        &
        \deduce{Y \mid \Gamma \vdash_{s} C}{}
      }
    \end{array}
  \end{displaymath}
  Structural and adjunction rules
  \begin{displaymath}
    \begin{array}{cc}
      % \infer[s\td\ex]{S_{s} \mid \Gamma_{0} , B , A , \Gamma_{1} \vdash_{s} C}{S_{s} \mid \Gamma_{0} , A , B , \Gamma_{1} \vdash_{s} C}
      \infer[s\td\wk]{S \mid  \Gamma , \Phi \vdash_{s} C}{S \mid \Gamma , X, \Phi \vdash_{s} C}
      \qquad
      \infer[s\td\ctr]{S \mid \Gamma , X, \Phi \vdash_{s} C}{S \mid \Gamma , X  ,X , \Phi \vdash_{s} C}
      \\
      \infer[\gl_{stp}]{GA \mid \Gamma , \Phi \vdash_{s} C}{A \mid \Gamma , \Phi \vdash_{s} C}
      \qquad
      \infer[\fl_{stp}]{FX \mid \Gamma , \Phi \vdash_{s} C}{X \mid \Gamma , \Phi \vdash_{s} C}
      \qquad
      \infer[\gr]{- \mid \Phi \vdash_{i} GA}{- \mid \Phi \vdash_{s} A}
    \end{array}
  \end{displaymath}
  Admissible cut rules\footnote{We only have five cut rules here because it is impossible to have $\scut$ in intuitionistic logic (the stoup is always empty).}
  \begin{displaymath}
    \begin{array}{cc}
      \infer[i\td\mathsf{cut}]{- \mid \Phi_{0} , \Phi_{1} , \Phi_{2} \vdash_{i} Z}{
       \deduce{- \mid \Phi_{1} \vdash_{i} X}{}
       &
       \deduce{- \mid \Phi_{0} , X , \Phi_{2} \vdash_{i} Z}{}
      }
       \\
       \infer[s\td\scut]{S \mid \Gamma_{0} , \Gamma_{1} \vdash_{s} C}{
        \deduce{S \mid \Gamma_{0} \vdash_{s} A}{}
        &
        \deduce{A \mid \Gamma_{1} \vdash_{s} C}{}
       }
       \qquad
       \infer[s\td\ccut]{S \mid \Gamma_{0} , \Gamma_{1} , \Gamma_{2} \vdash_{ss} C}{
        \deduce{- \mid \Gamma_{1} \vdash_{s} A}{}
        &
        \deduce{S \mid \Gamma_{0} , A , \Gamma_{1} \vdash_{s} C}{}
      }
      \\
      \infer[i\td s \td\scut]{- \mid \Phi , \Gamma \vdash_{s} C}{
       \deduce{- \mid \Phi \vdash_{i} X}{}
       &
       \deduce{X \mid \Gamma \vdash_{s} C}{}
      }
      \qquad
      \infer[i\td s\td\ccut]{S \mid \Gamma_{0} , \Phi , \Gamma_{1} \vdash_{s} C}{
       \deduce{- \mid \Phi \vdash_{i} X}{}
       &
       \deduce{S \mid \Gamma_{0} , X , \Gamma_{1} \vdash_{s} C}{}
      }
    \end{array}
  \end{displaymath}
  \section*{Notes on Skew symmetric LNL (Cartesian and SkSMCC)}
  \subsection*{Proof theory}
  The proof system here is cut-free obviously, then the next step is to consider the soundness and completeness with adjunction model and also focusing strategy in LNL system.
  Focusing for LNL is not yet developed thoroughly in literature, so I think it is a good opportunity to have a series of studies on different LNL logic and if we could, extend the method to adjoint logics.
  For now I know, only two papers by K. Pruiksma (2018, 2021) about focused system for adjoint logic.\footnote{K. Pruiksma is a PhD student in the Department of Computer Science, Carnegie Mellon University}

  \section*{Skew LNL (Cartesian and Skew monoidal)}
  Formulae and context in skew LNL:
  \begin{itemize}
    \item Intuitionistic part: $X \mid 1 \mid X \times Y \mid X \rightarrow Y \mid GA$, contexts $\Phi$
    \item Skew part: $A \mid \I \mid A \ot B \mid A \lolli B \mid FX$, contexts $\Gamma$. ($\Gamma$ is a mixed context containing Cartesian and non-Cartesian-formulae).
    \item There is no stoup formula in intuitionistic part of LNL.
  \end{itemize}
  Sequent calculus for intuitionistic LNL
  \begin{displaymath}
    \begin{array}{cc}
      \infer[i\td\ax]{- \mid A \vdash_{i} A}{}
      \qquad
      \infer[i\td 1\mathsf{R}]{- \mid \quad \vdash_{i} 1}{}
      \qquad
      \infer[\tmslf]{- \mid A \tms B , \Phi \vdash_{i} Z}{- \mid A,  \Phi \vdash_{i} Z}
      \qquad
      \infer[\tmsls]{- \mid A \tms B , \Phi \vdash_{i} Z}{- \mid B,  \Phi \vdash_{i} Z}
      \\
      \infer[\tmsr]{- \mid \Phi_{0} , \Phi_{1} \vdash_{i} A \tms B}{
       \deduce{- \mid \Phi_{0} \vdash_{i} X}{}
       &
       \deduce{- \mid \Phi_{1} \vdash_{i} Y}{}
      }
      \qquad
      \infer[\rarl]{- \mid X \rar Y , \Phi_{0} , \Phi_{1} \vdash_{i} Z}{
       \deduce{- \mid \Phi_{0} \vdash_{i} X}{}
       &
       \deduce{- \mid Y , \Phi_{1} \vdash_{i} Z}{}
      }
      \qquad
      \infer[\rarr]{- \mid \Phi \vdash_{i} X \rar Y}{- \mid \Phi , X \vdash_{i} Y}
    \end{array}
  \end{displaymath}
  Structural and adjunction rules\footnote{We do not need write out $\ex$ explicitly in LNL because formulae in context are always exchangeable. However, if we want to directly see adjunction between intuitionistic logic and skew non-comuutative linear logic, we have to write it explicitly.}
  \begin{displaymath}
    \begin{array}{cc}
    \infer[i\td\wk]{- \mid X, \Phi \vdash_{i} Z}{- \mid \Phi \vdash_{i} Z}
    \qquad
    \infer[i\td\ctr]{- \mid X , \Phi \vdash_{i} Z}{- \mid X , X , \Phi \vdash_{i} Z}
    \qquad
    % \infer[i\td\ex]{- \mid \Phi , Y , X \vdash_{i} Z}{- \mid \Phi , X , Y \vdash_{i} Z}
    % \qquad
    \infer[\fr]{- \mid \Phi \vdash_{s} F X}{- \mid \Phi \vdash_{i} X}
    \end{array}
  \end{displaymath}
  Sequent calculus for skew monoidal closed
  \vspace{-0.15cm}
   \begin{displaymath}
   \begin{array}{cc}
     \infer[sk\td\ax]{A \mid \quad \vdash_{sk} A}{}
     \qquad
     \infer[sk\td\unitr]{- \mid \quad \vdash_{sk} \I}{}
     \qquad
     \infer[sk\td\unitl]{\I \mid \Gamma \vdash_{sk} C}{- \mid \Gamma \vdash_{sk} C}
     \qquad
     \infer[sk\td\tl]{A \ot B \mid \Gamma \vdash_{sk} C}{A \mid B , \Gamma \vdash_{sk} C}
     \\
     \infer[sk\td\pass]{- \mid A, \Gamma \vdash_{sk} C}{A \mid \Gamma \vdash_{sk} C}
     \qquad
     \infer[sk\td\tr]{S \mid \Gamma_{0} , \Gamma_{1} \vdash_{sk} A \ot B}{
      \deduce{S_{s} \mid \Gamma_{0} \vdash_{sk} A}{}
      \qquad
      \deduce{- \mid \Gamma_{1} \vdash_{sk} B}{}
     }
     \qquad
     \infer[sk\td\lright]{S \mid \Gamma \vdash_{sk} A \lolli B}{S \mid \Gamma , A \vdash_{sk} B}
     \\
     \infer[sk\td\lleft]{A \lolli B \mid \Gamma_{0} , \Gamma_{1} \vdash_{sk} C}{
      \deduce{- \mid \Gamma_{0} \vdash_{sk} A}{}
      \qquad
      \deduce{B \mid \Gamma_{1} \vdash_{sk} C}{}
     }
   \end{array}
   \end{displaymath}
   Intuitionistic rules in skew monoidal closed calculus
   \begin{displaymath}
     \begin{array}{cc}
       \infer[sk\td\tmslf]{X \tms Y \mid \Gamma \vdash_{sk} C}{X \mid \Gamma \vdash_{sk} C}
       \qquad
       \infer[sk\td\tmsls]{X \tms Y \mid \Gamma \vdash_{sk} C}{Y \mid \Gamma \vdash_{sk} C}
       \qquad
       \infer[sk\td\rarl]{X \rightarrow Y \mid \Phi , \Gamma \vdash_{sk} C}{
         \deduce{- \mid \Phi \vdash_{i} X}{}
         &
         \deduce{Y \mid \Gamma \vdash_{sk} C}{}
       }
     \end{array}
   \end{displaymath}
   Structural and adjunction rules
   \begin{displaymath}
     \begin{array}{cc}
        \infer[sk\td\ex]{S \mid \Gamma_{0} , Y , X , \Gamma_{1} \vdash_{sk} C}{S \mid \Gamma_{0} , X , Y , \Gamma_{1} \vdash_{sk} C}
        \qquad
        \infer[sk\td\wk]{S \mid  \Gamma_{0} , X , \Gamma_{1} \vdash_{sk} C}{S \mid \Gamma_{0} , \Gamma_{1} \vdash_{sk} C}
        \qquad
        \infer[sk\td\ctr]{S \mid \Gamma_{0} , X, \Gamma_{1}, \Gamma_{2} \vdash_{sk} C}{S \mid \Gamma_{0} , X , \Gamma_{1} ,X, \Gamma_{2} \vdash_{sk} C}
        \\
       \infer[\gl_{stp}]{GA \mid \Gamma \vdash_{sk} C}{A \mid \Gamma \vdash_{sk} C}
       \qquad
       \infer[\fl_{stp}]{FX \mid \Gamma \vdash_{sk} C}{X \mid \Gamma \vdash_{sk} C}
       \qquad
       \infer[\gr]{- \mid \Phi \vdash_{i} GA}{- \mid \Phi \vdash_{sk} A}
     \end{array}
   \end{displaymath}
   Admissible cut rules\footnote{We only have five cut rules here because it is impossible to have $\scut$ in intuitionistic logic (the stoup is always empty).}
   \begin{displaymath}
     \begin{array}{cc}
       \infer[i\td\mathsf{cut}]{- \mid \Phi_{0} , \Phi_{1} , \Phi_{2} \vdash_{i} Z}{
        \deduce{- \mid \Phi_{1} \vdash_{i} X}{}
        &
        \deduce{- \mid \Phi_{0} , X , \Phi_{2} \vdash_{i} Z}{}
       }
        \\
        \infer[sk\td\scut]{S \mid \Gamma_{0} , \Gamma_{1} \vdash_{sk} C}{
         \deduce{S \mid \Gamma_{0} \vdash_{sk} A}{}
         &
         \deduce{A \mid \Gamma_{1} \vdash_{sk} C}{}
        }
        \qquad
        \infer[sk\td\ccut]{S \mid \Gamma_{0} , \Gamma_{1} , \Gamma_{2} \vdash_{sk} C}{
         \deduce{- \mid \Gamma_{1} \vdash_{sk} A}{}
         &
         \deduce{S \mid \Gamma_{0} , A , \Gamma_{1} \vdash_{sk} C}{}
       }
       \\
       \infer[i\td sk \td\scut]{- \mid \Phi , \Gamma \vdash_{sk} C}{
        \deduce{- \mid \Phi \vdash_{i} X}{}
        &
        \deduce{X \mid \Gamma \vdash_{sk} C}{}
       }
       \qquad
       \infer[i\td sk\td\ccut]{S \mid \Gamma_{0} , \Phi , \Gamma_{1} \vdash_{sk} C}{
        \deduce{- \mid \Phi \vdash_{i} X}{}
        &
        \deduce{S \mid \Gamma_{0} , X , \Gamma_{1} \vdash_{sk} C}{}
       }
     \end{array}
   \end{displaymath}
  \section*{Notes on Skew LNL (Cartesian and SkMCC)}
  \subsection*{Proof theory}
  If we have contraction in stoup in skew (symmetric) calculus
  \begin{displaymath}
    \infer[\mathsf{ctr}_{stp}]{X \mid \Gamma_{0} , \Gamma_{1} \vdash_{sk} C}{X \mid \Gamma_{0} , X, \Gamma_{1} \vdash_{sk} C}
  \end{displaymath}
  then we cannot prove the admissibility of $\ctr_{cxt}$
  \begin{displaymath}
    \infer[\ctr_{cxt}]{S \mid \Gamma_{0} , X , \Gamma_{1} , \Gamma_{2} \vdash_{sk} C}{S \mid \Gamma_{0} , X , \Gamma_{1} , X , \Gamma_{2} \vdash_{sk} C}
  \end{displaymath}
  We fail to prove it when the previous rule is a two-premises rule and contractum are in different premises.
  Therefore, it seems like we cannot have strutural rules in stoup position, but I don't know if this makes sense.
  We can observe that stoup in intuitionistic logic is always empty so it might be natural to think that structural rules can only active in context.

  Context versions of $sk\td \tmslf$, $sk\td \tmsls$, and $sk\td\rarl$ are admissible.
  \begin{displaymath}
    \begin{array}{cc}
      \infer[sk\td\tmslf_{cxt}]{S \mid \Gamma_{0} , X \tms Y , \Gamma_{1} \vdash_{sk} C}{S \mid \Gamma_{0} , X , \Gamma_{1} \vdash_{sk} C}
      \qquad
      \infer[sk\td\tmsls_{cxt}]{S \mid \Gamma_{0} , X \tms Y , \Gamma_{1} \vdash_{sk} C}{S \mid \Gamma_{0} , Y , \Gamma_{1} \vdash_{sk} C}
      \\
      \infer[sk\td\rarl_{cxt}]{S \mid \Phi , \Gamma_{0} , X \rar Y , \Gamma_{1} \vdash_{sk} C}{
        \deduce{- \mid \Phi \vdash_{i} X}{}
        &
        \deduce{S \mid \Gamma_{0} , Y , \Gamma_{1} \vdash_{sk} C}{}
      }
    \end{array}
  \end{displaymath}

  \subsection*{Motivation}
  Cartesian structure plays important role in semantics of logics, especially in interpreting structural rules.
  For example, categorical model of intuitionistic logic is Cartesian closed category, and one categorical model of intuitionistic multipicative linear logic is linear category which is a symmetric monoidal closed category equipped with symmetric monoidal comonad and its coEilenberg-Xoore category is Cartesian.
  In particular, diagonal morphism $\Delta : A \longrightarrow A\times A$ models contraction and $\tau_A : A \longrightarrow 1$ models weakening.
  Cartesian structure itself is implicitly associative and symmetric, we can see classic proof of these statements in category theory textbooks.

  In the last decade, skew monoidal categories come into discussion, and there are some studies using proof theoretical method to analyze categorical properties.
  These proof systems are noncommutative intuitionistic multiplicative linear logic (NMILL, henceforth).
  Beside noncommutative, they are non-associative.
  The leftmost position in the antecedent of whole sequent is called stoup which can be empty.
  We can passiviate stoup formula to context but cannont put it back from the other direction.

  We already have a sequent calculus system NMILLs which has classic linear logic connectives $\ot , \lolli$, and $\I$.
  It is natural to think about if we can add structural rules in NMILLs.
  At the first, we choose to directly add structural rules such as dereiliction, promotion, weakening, and contraction.
  However, we fail to prove the cut elimination due to lacking of associativity.
  In particular, when we have left premise is $\tl$ and the right premise is promotion, lacking of associativity causes impossibility to permute $\tl$ down.
  Surpisingly, LNL systax proposed by Benton et al. gives us a perfect solution.
  In LNL framework, we can have a cut free sequent calculus system.
  It shows that LNL framework provides a good environment to incorporate non-associative logic with Cartesian structures.
\end{document}
